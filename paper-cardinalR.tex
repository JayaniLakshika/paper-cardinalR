% !TeX root = RJwrapper.tex
\title{cardinalR: Generating interesting high-dimensional data structures}


\author{by Jayani P. Gamage, Dianne Cook, Paul Harrison, Michael Lydeamore, and Thiyanga S. Talagala}

\maketitle

\abstract{%
A high-dimensional dataset is where each observation is described by many features, or dimensions. Such a dataset might contain various types of structures that have complex geometric properties, such as nonlinear manifolds, clusters, or sparse distributions. We can generate data containing a variety of structures using mathematical functions and statistical distributions. Sampling from a multivariate normal distribution will generate data in an elliptical shape. Using a trigonometric function we can generate a spiral. A torus function can create a donut shape. High-dimensional data structures are useful for testing, validating, and improving algorithms used in dimensionality reduction, clustering, machine learning, and visualization. Their controlled complexity allows researchers to understand challenges posed in data analysis and helps to develop robust analytical methods across diverse scientific fields like bioinformatics, machine learning, and forensic science. Functions to generate a large variety of structures in high dimensions are organized into the R package \texttt{cardinalR}, along with some already generated examples.
}

\section{Introduction}\label{introduction}

Generating synthetic datasets with clearly defined geometric properties is essential for evaluating and benchmarking algorithms in various fields, such as machine learning, data mining, and computational biology. Researchers often need to generate data with specific dimensions, noise characteristics, and complex underlying structures to test the performance and robustness of their methods.

There are numerous packages available in R for generating synthetic data, each designed with unique characteristics and focus areas. For example, \texttt{geozoo} (\citet{barret2016}) offers a large collection of geometric objects, allowing users to create and analyze specific shapes, primarily in lower-dimensional spaces. Another useful package is \texttt{snedata} (\citet{james2025}), which provides tools for generating simplified datasets useful for evaluating dimensionality reduction techniques like tSNE, often focusing on understanding and evaluating low-dimensional embeddings of complex data structures. Additionally, \texttt{mlbench} (\citet{friedrich2024}) includes a collection of well-known benchmark datasets commonly associated with established classification or regression challenges. In the field of single-cell omics, \texttt{splatter} (\citet{luke2017}) is particularly simulate complex biological data, effectively capturing nuances such as batch effects and differential expression.

There is a valuable opportunity to improve the generation of high-dimensional data structures by integrating geometric principles with advanced noise control and customizable clustering. The \texttt{geozoo} package provides a strong foundation but could be enhanced to support high-dimensional extensions with controlled noise and user-defined parameters for clustering. Similarly, while \texttt{snedata} focuses on abstract datasets for dimensionality reduction, adding features for generating high-dimensional data from geometric layouts would enhance its usability. The \texttt{mlbench} package could also benefit from allowing users to create datasets with specific geometric structures and noise profiles. Additionally, although \texttt{splatter} specializes in biological data simulation, it could be expanded to offer a broader framework for generating diverse geometric structures across dimensions, enabling detailed control over noise and clustering. Addressing these areas could lead to more robust high-dimensional data generation tools.

To address this gap, this paper introduces the \texttt{cardinalR} R package. This package provides a collection of functions designed to generate customizable data structures in any number of dimensions, starting from basic geometric shapes. \texttt{cardinalR} offers important functionalities that extend beyond the capabilities of existing tools, allowing users to: (i) construct high-dimensional datasets based on geometric shapes, including the option to enhance dimensionality by adding controlled noise dimensions; (ii) introduce adjustable levels of background noise to these structures; and (iii) create complex clustered data arrangements by using fundamental geometric forms, while maintaining their positions, scales, orientations, and sample sizes in arbitrary dimensional spaces. By providing these integrated features, \texttt{cardinalR} aims to provide researchers to generate more explainable and challenging synthetic datasets focused to the specific needs of evaluating algorithms in high-dimensions. This bridges the gap between geometric foundations and the flexible generation of complex synthetic data.

The paper is organized as follows. In next section, introduces the implementation of \texttt{cardinalR} package on CRAN and GitHub, including demonstration of the package's key functions. We illustrate how a clustering data structure affect the dimension reductions in \textbf{Application} section. Finally, we give a brief conclusion of the paper and discuss potential opportunities for use of our data collection.

\section{Implementation}\label{implementation}

\subsection{Installation}\label{installation}

The package can be installed from CRAN using

\begin{verbatim}
install.packages("cardinalR")
\end{verbatim}

and the development version can be installed from GitHub

\begin{verbatim}
devtools::install_github("JayaniLakshika/cardinalR")
\end{verbatim}

\subsection{Web site}\label{web-site}

More documentation of the package can be found at the web site \url{https://jayanilakshika.github.io/cardinalR/}.

\subsection{Data sets}\label{data-sets}

The \texttt{cardinalR} package comes with several data sets that load with the package. These are described in Table \ref{tab:datasets-tb-pdf}.

\begin{table}

\caption{\label{tab:datasets-tb-pdf}cardinalR data sets}
\centering
\begin{tabular}[t]{>{\raggedright\arraybackslash}p{4cm}>{\raggedright\arraybackslash}p{8cm}}
\toprule
data & explanation\\
\midrule
mobiusgau & Simulated data with a Mobius and a Gaussian in 4-D space.\\
mobiusgau\_tsne1 & The tSNE embedding with perplexity \$15\$ for mobiusgau.\\
mobiusgau\_tsne2 & The tSNE embedding with perplexity \$30\$ for mobiusgau.\\
mobiusgau\_tsne3 & The tSNE embedding with perplexity \$5\$ for mobiusgau.\\
mobiusgau\_umap1 & The UMAP embedding with number of neighbors \$15\$ and minimum distance \$0.1\$ for mobiusgau.\\
\addlinespace
mobiusgau\_umap2 & The UMAP embedding with number of neighbors \$30\$ and minimum distance \$0.08\$ for mobiusgau.\\
mobiusgau\_umap3 & The UMAP embedding with number of neighbors \$5\$ and minimum distance \$0.9\$ for mobiusgau.\\
\bottomrule
\end{tabular}
\end{table}

\subsection{Functions}\label{functions}

\subsubsection{Main function}\label{main-function}

The main function of the package is \texttt{gen\_multicluster()}. This function generates clusters of various shapes, allowing users to specify the number of points in each cluster, as well as their locations, scaling, and rotations across specific dimensions. Additionally, users can add background noise into the generated data by using the \texttt{is\_bkg} option.

The main arguments of the \texttt{gen\_multicluster()} function are shown in Table \ref{tab:main-tb-pdf}.

\begin{table}

\caption{\label{tab:main-tb-pdf}The main arguments for gen\_multicluster().}
\centering
\begin{tabular}[t]{>{\raggedright\arraybackslash}p{4cm}>{\raggedright\arraybackslash}p{8cm}}
\toprule
Argument & Explanation\\
\midrule
n & A numeric vector representing the number of points in each cluster.\\
p & A numeric value representing the number of dimensions.\\
k & A numeric value representing the number of clusters.\\
loc & A numeric matrix representing the locations/centroids of clusters.\\
scale & A numeric vector representing the scaling factors of clusters.\\
\addlinespace
shape & A character vector representing the shapes of clusters.\\
rotation & A numeric list which contains plane and the corresponding angle along that plane for each cluster.\\
is\_bkg & A Boolean value representing the background noise should exist or not.\\
\bottomrule
\end{tabular}
\end{table}

\subsubsection{Branching}\label{branching}

A branching structure captures trajectories that diverge or bifurcate from a common origin, resembling processes such as cell differentiation in biology. We introduce a set of data generation functions specifically designed to simulate high-dimensional branching structures with various geometry, number of points, and the number of branches. Table \ref{tab:branching-tb-pdf} outlines these functions. The main arguments of the functions described in Table \ref{tab:arg-branching-tb-pdf}.

\begin{table}

\caption{\label{tab:branching-tb-pdf}cardinalR branching data generation functions}
\centering
\begin{tabular}[t]{>{\raggedright\arraybackslash}p{4cm}>{\raggedright\arraybackslash}p{8cm}}
\toprule
Function & Explanation\\
\midrule
gen\_expbranches & Generate a structure with exponential shaped branches.\\
gen\_linearbranches & Generate a structure with linear shaped branches.\\
gen\_curvybranches & Generate a structure with curvy shaped branches.\\
gen\_orglinearbranches & Generate a structure with linear shaped branches originated in one point.\\
gen\_orgcurvybranches & Generate a structure with curvy shaped branches originated in one point.\\
\bottomrule
\end{tabular}
\end{table}

\begin{table}

\caption{\label{tab:arg-branching-tb-pdf}The main arguments for branching shape generators.}
\centering
\begin{tabular}[t]{>{\raggedright\arraybackslash}p{4cm}>{\raggedright\arraybackslash}p{8cm}}
\toprule
Argument & Explanation\\
\midrule
n & A numeric value representing the number of points.\\
p & A numeric value representing the number of dimensions.\\
k & A numeric value representing the number of clusters.\\
\bottomrule
\end{tabular}
\end{table}

\subsubsection{Cone}\label{cone}

\begin{verbatim}
cone <- gen_cone(n = 1000, p = 4, h = 5, ratio = 0.5)
\end{verbatim}

\begin{figure}
\includegraphics[width=1\linewidth]{paper-cardinalR_files/figure-latex/fig-cone-proj-1} \caption{Three $2\text{-}D$ projections from $4\text{-}D$, for the `cone` data.}\label{fig:fig-cone-proj}
\end{figure}

\subsubsection{Cube}\label{cube}

\begin{table}

\caption{\label{tab:cube-tb-pdf}cardinalR cube data generation functions}
\centering
\begin{tabular}[t]{>{\raggedright\arraybackslash}p{4cm}>{\raggedright\arraybackslash}p{8cm}}
\toprule
Function & Explanation\\
\midrule
gen\_gridcube & Generate a cube with specified grid points along each axes.\\
gen\_unifcube & Generate a cube with uniform points.\\
gen\_cubehole & Generate a cube with a hole.\\
\bottomrule
\end{tabular}
\end{table}

\subsubsection{Gaussian}\label{gaussian}

\subsubsection{Linear}\label{linear}

\subsubsection{Mobius}\label{mobius}

\subsubsection{Polynomial}\label{polynomial}

\begin{table}

\caption{\label{tab:polynomial-tb-pdf}cardinalR polynomial data generation functions}
\centering
\begin{tabular}[t]{>{\raggedright\arraybackslash}p{4cm}>{\raggedright\arraybackslash}p{8cm}}
\toprule
Function & Explanation\\
\midrule
gen\_quadratic & Generate a quadratic pattern.\\
gen\_cubic & Generate a cubic pattern.\\
\bottomrule
\end{tabular}
\end{table}

\subsubsection{Pyramid}\label{pyramid}

\begin{table}

\caption{\label{tab:pyramid-tb-pdf}cardinalR pyramid data generation functions}
\centering
\begin{tabular}[t]{>{\raggedright\arraybackslash}p{4cm}>{\raggedright\arraybackslash}p{8cm}}
\toprule
Function & Explanation\\
\midrule
gen\_pyrrect & Generate a pyramid with a rectangular base, with the option of a sharp or blunted apex.\\
gen\_pyrtri & Generate a pyramid with a triangular base, with the option of a sharp or blunted apex.\\
gen\_pyrstar & Generate a pyramid with a star-shape base, with the option of a sharp or blunted apex.\\
gen\_pyrholes & Generate a pyramid with triangular pyramid shaped holes.\\
\bottomrule
\end{tabular}
\end{table}

\subsubsection{S-curve}\label{s-curve}

\begin{table}

\caption{\label{tab:scurve-tb-pdf}cardinalR S-curve data generation functions}
\centering
\begin{tabular}[t]{>{\raggedright\arraybackslash}p{4cm}>{\raggedright\arraybackslash}p{8cm}}
\toprule
Function & Explanation\\
\midrule
gen\_scurve & Generate a S-curve.\\
gen\_scurvehole & Generate a S-curve with a hole.\\
\bottomrule
\end{tabular}
\end{table}

\subsubsection{Sphere}\label{sphere}

\begin{table}

\caption{\label{tab:sphere-tb-pdf}cardinalR sphere data generation functions}
\centering
\begin{tabular}[t]{>{\raggedright\arraybackslash}p{4cm}>{\raggedright\arraybackslash}p{8cm}}
\toprule
Function & Explanation\\
\midrule
gen\_circle & Generate a circle.\\
gen\_curvycycle & Generate a curvy cell cycle.\\
gen\_unifsphere & Generate a uniform sphere.\\
gen\_gridedsphere & Generate a grided sphere.\\
gen\_clusteredspheres & Generate multiple small spheres within a big sphere.\\
\addlinespace
gen\_hemisphere & Generate a hemisphere.\\
\bottomrule
\end{tabular}
\end{table}

\subsubsection{Swiss Roll}\label{swiss-roll}

To generalize the Swiss roll structure to arbitrary dimensions, we introduce a function \texttt{generate\_swiss\_roll(n,\ p)}, which constructs a high-dimensional version of the classic 3D Swiss roll while preserving its core characteristics.

The function generates \texttt{n} points in a \texttt{p}-dimensional space, where the first two dimensions (\texttt{X\_1,\ X\_2}) define the primary Swiss roll shape using a parametric equation:

\[
X_1 = t \cos(t), \quad X_2 = t \sin(t), \quad \text{where } t \sim U(0, 3\pi)
\]

The third dimension (\texttt{X\_3}) introduces variation perpendicular to the roll, sampled uniformly from \([-1,1]\). Additional dimensions (\texttt{X\_4} to \texttt{X\_p}) extend the data structure by applying a \textbf{sinusoidal transformation} of the parameter \texttt{t}, ensuring continuity in higher-dimensional spaces:

\[
X_i = \frac{\sin(i t)}{i}, \quad \text{for } i \geq 4.
\]

This transformation ensures a gradual decay in variance across dimensions, mimicking real-world high-dimensional structures where later dimensions often capture subtler variations.

\subsubsection{Trigonometric}\label{trigonometric}

\begin{table}

\caption{\label{tab:trigonometric-tb-pdf}cardinalR trigonometric data generation functions}
\centering
\begin{tabular}[t]{>{\raggedright\arraybackslash}p{4cm}>{\raggedright\arraybackslash}p{8cm}}
\toprule
Function & Explanation\\
\midrule
gen\_crescent & Generate a crescent pattern.\\
gen\_curvycylinder & Generate a curvy cylinder.\\
gen\_sphericalspiral & Generate a spherical spiral.\\
gen\_helicalspiral & Generate a helical spiral.\\
gen\_conicspiral & Generate a conic spiral.\\
\addlinespace
gen\_nonlinear & Generate a nonlinear hyperbola.\\
\bottomrule
\end{tabular}
\end{table}

\subsubsection{Multiple cluster examples}\label{multiple-cluster-examples}

By using the shape generators mentioned above, we can create various examples of multiple clusters. The package includes some of these examples, which are described in Table \ref{tab:odd-shape-tb-pdf}.

\begin{table}

\caption{\label{tab:odd-shape-tb-pdf}cardinalR multiple clusters generation functions}
\centering
\begin{tabular}[t]{>{\raggedright\arraybackslash}p{4cm}>{\raggedright\arraybackslash}p{8cm}}
\toprule
Function & Explanation\\
\midrule
make\_mobiusgau & \\
make\_multigau & \\
\bottomrule
\end{tabular}
\end{table}

\subsubsection{Additional functions}\label{additional-functions}

The package includes various supplementary tools in addition to the shape-generating functions mentioned earlier. These tools allow users to generate noise dimensions with a normal distribution and various wavy patterns, create background noise, randomize the rows of the data, reposition clusters, generate a vector whose product and sum are approximately equal to a target value, rotate structures, and normalize the data. Table \ref{tab:add-tb-pdf} details these functions.

\begin{table}

\caption{\label{tab:add-tb-pdf}cardinalR additional functions}
\centering
\begin{tabular}[t]{>{\raggedright\arraybackslash}p{4cm}>{\raggedright\arraybackslash}p{8cm}}
\toprule
Function & Explanation\\
\midrule
gen\_noisedims & Generates additional noise dimensions.\\
gen\_bkgnoise & Adds background noise.\\
randomize\_rows & Randomizes the rows.\\
relocate\_clusters & Relocates the clusters.\\
gen\_nproduct & Generates a vector of positive integers whose product is approximately equal to a target value.\\
\addlinespace
gen\_nsum & Generates a vector of positive integers whose summation is approximately equal to a target value.\\
gen\_wavydims1 & Generates random noise dimensions with wavy pattern generated with theta.\\
gen\_wavydims2 & Generates random noise dimensions with wavy pattern generated with power functions.\\
gen\_wavydims3 & Generates random noise dimensions with wavy pattern generated with power and sine functions.\\
gen\_rotation & Generates rotations.\\
\addlinespace
normalize\_data & Normalizes data.\\
\bottomrule
\end{tabular}
\end{table}

\section{Application}\label{application}

\subsection{Assessing the performance of dimension reduction on different geometric structures in high-dimensions}\label{assessing-the-performance-of-dimension-reduction-on-different-geometric-structures-in-high-dimensions}

This section illustrates the use of package by generating a synthetic dataset to evaluate the performance of six popular dimension reduction techniques: Principal Component Analysis (PCA) \citep{jolliffe2011}, t-distributed stochastic neighbor embedding (tSNE) \citep{laurens2008}, uniform manifold approximation and projection (UMAP) \citep{leland2018}, potential of heat-diffusion for affinity-based trajectory embedding (PHATE) algorithm \citep{moon2019}, large-scale dimensionality reduction Using triplets (TriMAP) \citep{amid2019}, and pairwise controlled manifold approximation (PaCMAP) \citep{yingfan2021}.

The following code generates a dataset of five clusters, positioned with equal inter-cluster distances in \(4\text{-}D\) space (Figure \ref{fig:fig-highd-proj}).

\begin{figure}
\includegraphics[width=1\linewidth]{paper-cardinalR_files/figure-latex/fig-highd-proj-1} \caption{Three $2\text{-}D$ projections from $4\text{-}D$, for the `mobiusgau` data. The helical spiral cluster is represented in dark green, the hemisphere cluster in orange, the uniform cube-shaped cluster in purple, the blunted cone cluster in pink, and the Gaussian-shaped cluster in light green.}\label{fig:fig-highd-proj}
\end{figure}

The five clusters have different geometric structures and each contain different number of points. Specifically, the helical spiral cluster includes \(2250\) points and was generated with a scale parameter of \(0.4\). The hemisphere cluster consists of \(1500\) points with a scale parameter of \(0.35\). The uniform cube-shaped cluster contains \(750\) points and uses a scale parameter of \(0.3\). The blunted cone cluster includes \(1250\) points, generated with a scale parameter of \(1\). Finally, the Gaussian-shaped cluster contains \(1750\) points and was generated with a scale parameter of \(0.3\).

\begin{figure}
\includegraphics[width=1\linewidth]{paper-cardinalR_files/figure-latex/fig-nldr-layouts-1} \caption{Six different dimension reduction representations of the `mobiusgau` data using default hyperparameter settings: (a) tSNE, (b) UMAP, (c) PAHTE, (d) TriMAP, (e) PaCMAP, and (f) PCA.}\label{fig:fig-nldr-layouts}
\end{figure}

UMAP, PHATE, TriMAP, and PaCMAP effectively separate the five clusters and show the preservation of the global structure (Figure \ref{fig:fig-nldr-layouts}). However, PHATE reveals three non-linear clusters, even though two of them do not show non-linearity. UMAP, TriMAP, and PaCMAP successfully maintain the local structures of the data. In contrast, tSNE divides the non-linear cluster into sub-clusters. Also, tSNE fails to preserve the distances between the clusters. PCA, on the other hand, preserves the local structures of the clusters, but some clusters are incorrectly merged that should remain distinct.

\section{Discussion}\label{discussion}

\begin{itemize}
\tightlist
\item
  Branching: These functions create a controlled environment for testing how effectively various algorithms preserve branching topology and continuity in their low-dimensional embeddings.
\end{itemize}

\section{Code}\label{code}

The code is available at \url{https://github.com/JayaniLakshika/cardinalR}, and source material for this paper is available at \url{https://github.com/JayaniLakshika/paper-cardinalR}.

\section{Acknowledgements}\label{acknowledgements}

This article is created using \CRANpkg{knitr} \citep{yihui2015} and \CRANpkg{rmarkdown} \citep{yihui2018} in R with the \texttt{rjtools::rjournal\_article} template.

\bibliography{paper-cardinalR.bib}

\address{%
Jayani P. Gamage\\
Monash University\\%
Department of Econometrics and Business Statistics, VIC 3800 Australia\\
%
\url{https://jayanilakshika.netlify.app/}\\%
\textit{ORCiD: \href{https://orcid.org/0000-0002-6265-6481}{0000-0002-6265-6481}}\\%
\email{jayani.piyadigamage@monash.edu}%
}

\address{%
Dianne Cook\\
Monash University\\%
Department of Econometrics and Business Statistics, VIC 3800 Australia\\
%
\url{http://www.dicook.org/}\\%
\textit{ORCiD: \href{https://orcid.org/0000-0002-3813-7155}{0000-0002-3813-7155}}\\%
\href{mailto:dicook@monash.edu}{\nolinkurl{dicook@monash.edu}}%
}

\address{%
Paul Harrison\\
Monash University\\%
MGBP, BDInstitute, VIC 3800 Australia\\
%
%
\textit{ORCiD: \href{https://orcid.org/0000-0002-3980-268X}{0000-0002-3980-268X}}\\%
\href{mailto:paul.harrison@monash.edu}{\nolinkurl{paul.harrison@monash.edu}}%
}

\address{%
Michael Lydeamore\\
Monash University\\%
Department of Econometrics and Business Statistics, VIC 3800 Australia\\
%
%
\textit{ORCiD: \href{https://orcid.org/0000-0001-6515-827X}{0000-0001-6515-827X}}\\%
\href{mailto:michael.lydeamore@monash.edu}{\nolinkurl{michael.lydeamore@monash.edu}}%
}

\address{%
Thiyanga S. Talagala\\
University of Sri Jayewardenepura\\%
Department of Statistics, Gangodawila, Nugegoda 10100 Sri Lanka\\
%
\url{https://thiyanga.netlify.app/}\\%
\textit{ORCiD: \href{https://orcid.org/0000-0002-0656-9789}{0000-0002-0656-9789}}\\%
\href{mailto:ttalagala@sjp.ac.lk}{\nolinkurl{ttalagala@sjp.ac.lk}}%
}
